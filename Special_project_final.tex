\documentclass[letterpaper,12pt]{article}
\usepackage{textcomp}
\usepackage{fancyhdr}
\usepackage{amsmath}
\usepackage{ chemfig}
\usepackage{titlesec}
\usepackage{wrapfig}
\usepackage{caption}
\usepackage{epstopdf}
\usepackage{epsfig}

%\usepackage{MOdiagram}

\captionsetup[figure]{singlelinecheck=false}
\captionsetup[table]{singlelinecheck=false}

\renewcommand{\thesection}

\renewcommand{\headrulewidth}{0pt}

\addtolength{\oddsidemargin}{-.5in}
	\addtolength{\evensidemargin}{-.5in}
	\addtolength{\textwidth}{1in}

	\addtolength{\topmargin}{-.75in}
	\addtolength{\textheight}{1.5in}

\linespread{1.5}

\fancyhf{}
\pdfpageheight 11in
\pdfpagewidth 8.5in
\pagestyle{fancy}
\rhead{Skeel \thepage}

\begin{document}
	\noindent
	Brighton Skeel \\ 
	Partner: Myron Liu \\
	Lab Section 113\\
	\begin{center}
		\large\textbf{Ligand/Solvent Pair Efficacy in Aqueous Heavy Metal Complex Removal from Mixed Matrices}
	\end{center}
	\vspace{-1cm}
	\begin{flushleft}
		\textbf{\thesection{Goal}}
	\end{flushleft}
	This experiment aims to evaluate the efficacy of various organic solvent/phase transfer agent pairs in removing heavy metals from various aqueous matrices. To maintain a real world applicability, this experiment will use matrices representing the following matrices: acidified, seawater, distilled water, and hard water.
	\vspace{-0.5cm}
	\begin{flushleft}
		\textbf{\thesection{Introduction}}
	\end{flushleft}
		Since the dawn of the industrial age, the human race has developed a particular dependence upon metals--both rare and common. As these metal resources deplete, we turn to sources of lessening quality, many of which contain unwanted and toxic by-products such as chromium, cadmium, nickel, lead, cobalt, copper, and zinc~\cite{c1}. Though care is taken to prevent these substances from infiltrating the environment, accidents do happen (think of the recent flint water crisis or recent mining accidents in South America). In absence of adequate preventative methods, we are forced to consider retroactive methods. One such method, explored in this experiment is the use of phase transfer agents and organic solvents to complex heavy metals and physically extract them from contaminated water. Central to this process is the tendency of transition metals to preferentially form bonds with certain molecules (ligands) over water. These metal-ligand complexes have, in certain cases, properties analogous to those of micell structures---that is, they are much more soluble in non-polar solvents, and thus easily extracted from water. Exploiting these properties carefully, we plan to complex heavy metals under various conditions, with the expectation that more complex solutions will degrade heavy metal percentage elimination due to competitive effects (the heavy metals will, in many cases, have to compete with other, substantially less toxic cations such as sodium and magnesium).
	
	\vspace{-0.5cm}
	\begin{flushleft}
		\textbf{\thesection{Experimental Procedure}}
	\end{flushleft}
	\Large\emph{Week One:}
	\tiny 
	%	\line(1,0){425}
\normalsize \\
	\textbf{Matrix:} Distilled Water\\
	\textbf{Solvent Assignments:} Brighton-Tetrachloroethylene; Myron-Hexanes\\
	\indent 
	Both members will begin by going to ISF to obtain all glassware, equipment, and chemicals to be used throughout the day. Both will then proceed by acid washing the glassware they plan to use throughout the course of the day. For the first day, this also includes the $500mL$ volumetic flask to be used for the preparation of the master heavy metal matrix solution. An outline of the procedure for acid washing is outlined below, as taked from the Chemistry 4B lab manual \cite{c2}.
	\begin{enumerate}
		\item Rinse with a small (~$3mL$) ammount of $50\%$ $HNO_3$
		\item Rinse three times with tap water
		\item Rinse with a small (~$3mL$) ammount of $10\%$ $HCl$
		\item Rinse three times with tap water
		\item Rinse two times with distilled water
	\end{enumerate}
	\newpage

	
	\begin{thebibliography}{99}
		\bibitem{c1} Doong, R.; Lee, S.; Sun, Y.; Wu, S. \emph{Marine Pollution Bulletin} 2008, $57$, 846-857
		\bibitem{c2} Chemistry Lab Development,. \emph{Quantitative Analysis of a Solution Containing Co and Cu, Part 1}; University of California, Berkeley, 2016.
	\end{thebibliography}


\end{document}