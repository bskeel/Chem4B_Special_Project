\documentclass[letterpaper,12pt]{article}
\usepackage{textcomp}
\usepackage{fancyhdr}
\usepackage{amsmath}
\usepackage{ chemfig}
\usepackage{titlesec}
\usepackage{wrapfig}
\usepackage{caption}
\usepackage{epstopdf}
\usepackage{epsfig}

%\usepackage{MOdiagram}

\captionsetup[figure]{singlelinecheck=false}
\captionsetup[table]{singlelinecheck=false}

\renewcommand{\thesection}

\renewcommand{\headrulewidth}{0pt}

\addtolength{\oddsidemargin}{-.5in}
	\addtolength{\evensidemargin}{-.5in}
	\addtolength{\textwidth}{1in}

	\addtolength{\topmargin}{-.75in}
	\addtolength{\textheight}{1.5in}

\linespread{1.5}

\fancyhf{}
\pdfpageheight 11in
\pdfpagewidth 8.5in
\pagestyle{fancy}
\rhead{Skeel \thepage}

\begin{document}
	\noindent
	Brighton Skeel \\ 
	Partner: Myron Liu \\
	Lab Section 113\\
	\begin{center}
		\large\textbf{Ligand/Solvent Pair Efficacy in Aqueous Heavy Metal Complex Removal from Mixed Matrices}
	\end{center}
	\vspace{-1cm}
	\begin{flushleft}
		\textbf{\thesection{Goal}}
	\end{flushleft}
	This experiment aims to evaluate the efficacy of various organic solvent/phase transfer agent pairs in removing heavy metals from various aqueous matrices. To maintain a real world applicability, this experiment will use matrices representing the following matrices: acidified, seawater, distilled water, and hard water.
	\vspace{-0.5cm}
	\begin{flushleft}
		\textbf{\thesection{Introduction}}
	\end{flushleft}
		Since the dawn of the industrial age, the human race has developed a particular dependence upon metals--both rare and common. As these metal resources deplete, we turn to sources of lessening quality, many of which contain unwanted and toxic by-products such as chromium, cadmium, nickel, lead, cobalt, copper, and zinc~\cite{c1}. Though care is taken to prevent these substances from infiltrating the environment, accidents do happen (think of the recent flint water crisis or recent mining accidents in South America). In absence of adequate preventative methods, we are forced to consider retroactive methods. One such method, explored in this experiment is the use of phase transfer agents and organic solvents to complex heavy metals and physically extract them from contaminated water. Central to this process is the tendency of transition metals to preferentially form bonds with certain molecules (ligands) over water. These metal-ligand complexes have, in certain cases, properties analogous to those of micell structures---that is, they are much more soluble in non-polar solvents, and thus easily extracted from water. Exploiting these properties carefully, we plan to complex heavy metals under various conditions, with the expectation that more complex solutions will degrade heavy metal percentage elimination due to competitive effects (the heavy metals will, in many cases, have to compete with other, substantially less toxic cations such as sodium and magnesium).
	
	\vspace{-0.5cm}
	\begin{flushleft}
		\textbf{\thesection{Experimental Procedure}}
	\end{flushleft}
	\Large\emph{Week One:}
	\tiny 
	%	\line(1,0){425}
\normalsize \\
	\textbf{Matrix:} Distilled Water\\
	\textbf{Solvent Assignments:} Brighton-Tetrachloroethylene; Myron-Hexanes\\
	\indent 
	Both members will begin by going to ISF to obtain all glassware, equipment, and chemicals to be used throughout the day. Both will then proceed by acid washing the glassware they plan to use throughout the course of the day. For the first day, this also includes the $500mL$ volumetric flask to be used for the preparation of the master heavy metal matrix solution. An outline of the procedure for acid washing is outlined below, as taken from the Chemistry 4B lab manual \cite{c2}.
	\begin{enumerate}
		\item Rinse with a small (~$3mL$) amount of $50\%$ $HNO_3$
		\item Rinse three times with tap water
		\item Rinse with a small (~$3mL$) amount of $10\%$ $HCl$
		\item Rinse three times with tap water
		\item Rinse two times with distilled water
	\end{enumerate}
	\indent
	Following an acid wash of all glassware, the master solution will be prepared. This experiment is based on the premise of $ppm$ metal concentrations, so metal salts will be added as such. To achieve appropriate metal concentrations (400-500ppm) for eventual dilution, we will add $0.2250(\pm0.0250)g$ of each of the following salts to the flask, mixing thoroughly:
	\begin{center}
		\begin{tabular}[htbp]{c@{ : }r}
			\emph{Cadmium Chloride} & $CdCl_2$ \\
			\emph{Cobalt (II) Chloride} & $CoCl_2$ \\
			\emph{Chromium (III) Chloride} & $CrCl_3$ \\
			\emph{Copper (II) Chloride} & $CuCl_2$ \\
			\emph{Lead (II) Chloride} & $PbCl_2$ \\
			\emph{Zinc (II) Chloride} & $ZnCl_2$ \\
		\end{tabular}
	\end{center}
	
	\indent
	This solution will be sample according to what, from here onward, we will refer to as the AES sample preparation procedure. The procedure is outlined below, and has been optimized to reduce fluid waste:
	\begin{enumerate}
		\item Sample $1mL$ of the matrix
		\item Dilute this sample in a $10mL$ volumetric flask to volume with $0.2\%$ trace metal grade $HNO_3$
		\item Dilute $5mL$ of this solution to $50mL$ in a $50mL$ volumetric flask using $0.2\%$ trace metal grade $HNO_3$. (Note that this results in a net dilution factor of $100\times$, which is an acceptable level for MP-AES analysis\cite{c2})
		\item Filter an aliquot of this dilute sample through a regenerated cellulose membrane and into a $50mL$ falcon tube for processing with MP-AES
		\item Label the falcon tube with the sample name, date, and any other relevant information.
	\end{enumerate}
	
	\indent
	After sampling the master heavy metal matrix, pH and temperature data for the solution will be recorded. Following these measurements a ~$100mL$ aliquot will be drawn off and split into two separate beakers. These beakers will be appropriately labeled, and will be used by each group member (I.e., each individual will receive one of these beakers). Remaining matrix solution will be stored in two $250mL$ \emph{plastic} bottles (note the use of plastic bottles for long term storage--this mitigates error inherent from ion adsorption onto glass surfaces).
	\\
	\indent
	Following matrix partitioning so as to supply solution to each member, the complexion portion of the experiment will be carried out. Each member begins by placing \emph{exactly} $5mL$ of their organic solvent into a $15mL$ plastic falcon tube (centrifuge style). \textbf{Safety Note:} As the fumes from organic solvents, particularly the tetrachloroethylene used in this experiment, are harmful, all solvents, when not in use, will be stored in sealed glass bottles. Falcon tubes containing solvent must be closed after delivery of solvent, so as to prevent excessive fuming. After addition of solvent, the complexing agent will be dissolved in the organic phase. So as to obtain meaningful data, the complexing agent will be added in \emph{sub}-stoichiometric quantities (so as to elucidate phase transfer agent affinity to certain metals). Regarding this step, we recognize that the three complexion agents used in this experiment tend to form differing numbers of bonds to metal atoms, as tabulated below.  
	\begin{center}
		\begin{tabular}[htbp]{c@{ : }l}
			\emph{Dithizone (dithiocarbazone)} & Bidentate, 2 $mol$ per complex (tetrahedral complexion)\\
			\emph{Bipy (2,2'-bipyridine)} & Bidentate, 3 $mol$ per complex (octahedral complexion)\\
			\emph{Dibenzo 24-crown-8} & Polydentate, 1-2 $mol$ per complex (cation-size dependent)
		\end{tabular}
	\end{center}
	
	\indent
	As we know the concentration of heavy metal ions in solution (assuming the use of $0.2250g/500mL$ for a total metal concentration of $2.7\times 10^{-3}\frac{g}{mL}$), we compute that the total cation molarity must be $1.7_0\times 10^{-5}\frac{mol}{mL}$. Using these data (assuming the crown ether will coordinate with, on average, $\frac{2}{3}$ of a cation), as well as the (arbitrarily assigned) sub-stoichiometric quantities of complexion agent to be $25\%,\text{ ,}50\%\text{, and }75\%$, we arrive at the following mass values for complexion agent addition to each solution:
	\begin{center}
		\begin{tabular}[htbp]{r|c|c|c}
			& \textbf{Dithizone} & \textbf{Bipy} & \textbf{24-crown-8}\\
			\hline
			$25\%$ & $10.88mg$ & $9.96mg$ & $14.30mg$\\
			\hline 
			$50\%$ & $21.76mg$ & $19.91mg$ & $28.59mg$\\
			\hline
			$75\%$ & $32.64mg$ & $29.87mg$ & $42.89mg$
		\end{tabular}
	\end{center}
	
	\indent
	After adding these complexion agents, the falcon tubes will be shaken to mix the solvent and complexion agent. After ~$5$ minutes of mixing, $5mL$ of metal solution will be added to each of the sample vials. These will, again, be mixed, though this time more gently so as to prevent the formation of an emulsion, which could jeopardize the experiment. At this point, excess organic solvent will be drawn off and disposed of properly. There should be no organic solvent remaining, so removal of some sample stock is acceptable, so long as \emph{at least} $1mL$ of sample solution remains. This sample solution will be prepared as described in the previous section (see AES preparation technique). By the last hour of lab period, each group member will have prepared nine samples for submission to ISF. These samples will be evaluated for heavy metal content only (in this case, cobalt, copper, chromium, cadmium, lead, and zinc). The last half-hour of class will be devoted to cleanup and effectively disposing of waste, as well as preparing for the next laboratory session.
	\newpage

	
	\begin{thebibliography}{99}
		\bibitem{c1} Doong, R.; Lee, S.; Sun, Y.; Wu, S. \emph{Marine Pollution Bulletin} 2008, $57$, 846-857
		\bibitem{c2} Chemistry Lab Development,. \emph{Quantitative Analysis of a Solution Containing Co and Cu, Part 1}; University of California, Berkeley, 2016.
	\end{thebibliography}


\end{document}